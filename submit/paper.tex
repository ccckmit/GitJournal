\documentclass{article}


\usepackage{../template/template}

\usepackage[utf8]{inputenc} % allow utf-8 input
\usepackage[T1]{fontenc}    % use 8-bit T1 fonts
\usepackage{hyperref}       % hyperlinks
\usepackage{url}            % simple URL typesetting
\usepackage{booktabs}       % professional-quality tables
\usepackage{amsfonts}       % blackboard math symbols
\usepackage{nicefrac}       % compact symbols for 1/2, etc.
\usepackage{microtype}      % microtypography
\usepackage{lipsum}

\providecommand{\tightlist}{%
  \setlength{\itemsep}{0pt}\setlength{\parskip}{0pt}}

\title{GitJournal: 基於 git 的期刊出版新模式}

\author{
      陳鍾誠 \\
    國立金門大學資訊工程系 \\
    \texttt{ccc@nqu.edu.tw} \\
    \And
  }

\begin{document}
\maketitle

\begin{abstract}
  本文提出一種基於 git 的學術投稿、期刊審查方法,該方法透過像 github
  這樣的網路平台,讓投稿者與期刊之間能夠更順暢運作,也讓除了論文以外的學術資源,像是《原始碼、資料、還有測試工具》等資源,能夠更有效地公開並接受檢驗。
\end{abstract}

% keywords can be removed
\keywords{
   Git \and  Github \and  Markdown \and  Latex \and  出版 \and 
}

\hypertarget{ux7c21ux4ecb}{%
\section{簡介}\label{ux7c21ux4ecb}}

傳統學術運作方法是基於《紙本》的生態圈,採用《期刊-投稿》的方式進行。但是在網路與
web
技術開始改變各種領域的運作方式之後,這樣的模式就顯現出一些缺點,例如:

\begin{enumerate}
\def\labelenumi{\arabic{enumi}.}
\tightlist
\item
  學術研究背後的《資料、程式、還有實驗工具》等通常不容易公開,研究者很難檢驗該方法的效果並進一步改良。
\item
  期刊出版商的權力過大,並且透過壟斷論文商業權來營利,未付費的使用者通常無法閱讀這些研究論文。
\item
  投稿的過程冗長,導致學術研究沒辦法及時有效的出版,這對學術的發展有不利的影響。
\end{enumerate}

為了改進這些缺點,本文提出一種基於 git
專案的新型學術出版模式,這種模式透過像 git 的 push, pull, merge
機制,以及像 github 這樣的 git 平台之 fork, organization, gh-pages
等機制,建立一個《學術出版的新流程》,以下我們將描述這個流程的運作方法,並且提供一組讓這個流程可以有效運作的工具。

由於這個《新流程》是運作在像 github
這個平台上的,為了能夠清楚闡述這個流程,我們首先回顧一下 github
在程式社群上的運作方法。

\hypertarget{git-github-ux8207ux7a0bux5f0fux793eux7fa4ux7684ux904bux4f5cux65b9ux6cd5}{%
\section{Git, Github
與程式社群的運作方法}\label{git-github-ux8207ux7a0bux5f0fux793eux7fa4ux7684ux904bux4f5cux65b9ux6cd5}}

在《網路興起與開放原始碼》兩者的結合之下,程式社群開始有了《透過網路管理並交流程式》的強大需求,於是各種《程式碼管理工具》被設計了出來,從早期的
RCS, CVS, SVN 集中式版本管理工具,到較新的 Mercurial 與 Git
等分散型版本管理工具,都是為了讓《程式碼能被有效管理》而設計的。

舉例而言,《git
版本管理系統》通常會記錄《每個檔案的各個歷史版本》,能夠讓程式開發者有效的協作,透過
clone
指令複製最新版本之後,就能進行修改與進一步開發,等到一個新版本開發完成之後,就能透過
add, commit, push
等指令將新版程式推回《版本管理系統》中,讓其他人可以取得新版本後進一步測試開發。

如果新版程式開發失敗,或者新方法不能達到預期結果,那麼也可以放棄新版,透過
checkout
回復到舊版本去,這樣就能夠有效的管理程式碼,並且逐步的改進程式專案。

在 Linus Tovalds 設計出 git 版本管理系統之後,很多程式設計者就開始利用
git 管理程式碼,但是《程式碼公開交流》的問題只靠 git
是不夠的,還需要有《網站與社群》的支持功能, github
的出現進一步促進的《程式碼的大規模公開交流》現象,讓《開放原始碼的社群》能夠更有效的交流運作。

Github 提供了 fork 與 pull request
機制,讓想要改進程式的人可以在不干擾原作者的情況之下,複製出另一份專案,然後開始修改,等到程式改進完畢,又可以透過
pull request
機制請求原作者將該改進功能合併回原專案中,這樣的機制可以促進程式領域的良性交流,這種作法又稱為《社交式編程》。

另外、在 github 上還可以對專案按下 Star 與
Watch,這可以讓程式開發者知道那些專案比較受人關注,這些功能都進一步促進了《開放原始碼》的交流行為。

有了上述理解之後,我們就可以詳細描述本文所要提出的《基於 git
的科學出版模式》了!

\hypertarget{ux57faux65bc-git-ux7684ux5b78ux8853ux51faux7248ux65b9ux6cd5}{%
\section{基於 git
的學術出版方法}\label{ux57faux65bc-git-ux7684ux5b78ux8853ux51faux7248ux65b9ux6cd5}}

首先讓我們看這個《git 出版模式》中,期刊出版者的運作流程。

期刊出版者首先創建一個 github 專案後,將《投稿規格檔》(通常是一個 latex
樣板)
放上該專案,裡面還附上一些專案相關資源的資料夾規劃,以及對應的出版工具程式等等。

投稿者可以 fork
該期刊的專案,然後將論文內容放入對應的檔案中,並且檢視看看論文的呈現有沒有錯誤,確認論文完成無誤之後,就可以發送
pull requrest 讓出版商來審查並收件。

除了論文之外,投稿者也可以一併附上《研究的資料、程式與工具》等等資源,放入專案當中!

期刊出版者收到該 pull request
之後,可以啟動審查機制,並發送邀請給審查者,審查者核可之後,期刊出版者就可以接受
pull request 完成投稿程序。

然後期刊出版者就能編輯、排版、並進行刊登,不管是要發行電子版或紙本都行。

\hypertarget{ux904bux4f5cux6d41ux7a0b-ux4ee5git-ux958bux6e90ux671fux520aux70baux4f8b}{%
\section{運作流程 -- 以《Git
開源期刊》為例}\label{ux904bux4f5cux6d41ux7a0b-ux4ee5git-ux958bux6e90ux671fux520aux70baux4f8b}}

期刊出版者收到該 pull request
之後,可以啟動審查機制,並發送邀請給審查者,審查者核可之後,期刊出版者就可以接受
pull request 完成投稿程序。

然後期刊出版者就能編輯、排版、並進行刊登,不管是要發行電子版或紙本都行

在 陳鍾誠 (2014) 中提出了一套 \ldots{} 方法

(陳鍾誠 2014).

\[
\int_0^{\infty} f(x) dx
\]

\hypertarget{ux53c3ux8003ux6587ux737b}{%
\section*{參考文獻}\label{ux53c3ux8003ux6587ux737b}}
\addcontentsline{toc}{section}{參考文獻}

\hypertarget{refs}{}
\leavevmode\hypertarget{ref-ccc2019}{}%
陳鍾誠. 2014. ``Github 的使用方法.'' In \emph{網路個人出版學}, 417--22.
Github.


\end{document}
